\chapter{Conclusion}

The development of DengueWatch marks a transformative leap forward in public health technology, providing Iloilo City with a centralized system to combat one of the most persistent mosquito-borne diseases. Previously, data was recorded manually on paper, making tracking and analysis slow and error-prone. DengueWatch digitizes this process, enabling faster, more accurate monitoring. More than an academic project, DengueWatch serves as a practical solution aimed at shifting the approach from reactive outbreak response to proactive prevention. By combining deep learning models with real-time climate data integration, the system achieves a level of accuracy and usability that makes it viable for real-world deployment.

At the heart of DengueWatch is a Long Short-Term Memory (LSTM) neural network, which outperformed traditional forecasting models such as ARIMA and Kalman Filter. The LSTM model achieved a Root Mean Square Error (RMSE) of 20.15, followed by the hybrid KF-LSTM model with an RMSE of 25.56, demonstrating a substantial improvement in predictive capability. Consequently, the LSTM model was selected for integration into the DengueWatch system. Retraining the model monthly strikes a balance between maintaining accuracy and managing computational costs. It allows the model to incorporate new trends from the latest four weeks of data and aligns with the typical monthly data release schedule of provincial health offices.


Usability testing further underscored DengueWatch’s readiness for real-world deployment. The system achieved an average System Usability Scale (SUS) score of 88.5, significantly above the industry benchmark of 68. This indicates that users found the system intuitive, efficient, and suitable for operational use in public health contexts. Key features such as a user-friendly dashboard and a two-week forecasting window ensure that the system supports timely, effective responses.

Beyond its immediate application in Iloilo City, the framework behind DengueWatch holds the potential for broader impact. With minor adaptations, it can be scaled nationally through integration with Department of Health surveillance systems.


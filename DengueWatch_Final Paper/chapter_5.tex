\chapter{Conclusion}

The development of DengueWatch marks a transformative leap forward in public health technology, providing Iloilo City with a centralized system to combat one of the most persistent mosquito-borne diseases. Previously, data was recorded manually on paper, making tracking and analysis slow and error-prone. DengueWatch digitizes this process, enabling faster, more accurate monitoring. More than an academic project, DengueWatch serves as a practical solution aimed at shifting the approach from reactive outbreak response to proactive prevention. By combining deep learning models with real-time climate data integration, the system achieves a level of accuracy and usability that makes it viable for real-world deployment.

At the heart of DengueWatch is a Long Short-Term Memory (LSTM) neural network, which outperformed traditional forecasting models such as ARIMA and Kalman Filter. The LSTM model achieved a Root Mean Square Error (RMSE) of 16.90, compared to 39.00 and 38.40 for ARIMA and Kalman, respectively—demonstrating a substantial improvement in predictive capability. This advantage stems from the LSTM's ability to capture long-term dependencies and model nonlinear relationships between environmental factors and disease patterns.

The analysis also revealed that climate indicators, particularly rainfall and humidity, play a significant role in dengue outbreaks, typically leading to a surge in cases three to five weeks after anomalies are detected. By incorporating these lagged effects, DengueWatch achieved an explanatory power of 83\% (R² = 0.83), offering a game-changing advantage for early intervention and resource allocation.

Usability testing further underscored DengueWatch’s readiness for real-world deployment. The system achieved an average System Usability Scale (SUS) score of 88.5, significantly above the industry benchmark of 68. This indicates that users found the system intuitive, efficient, and suitable for operational use in public health contexts. Key features such as a user-friendly dashboard, a two-week forecasting window aligned with mosquito life cycles, and automated outbreak alerts ensure that the system supports timely, effective responses.

Beyond its immediate application in Iloilo City, the framework behind DengueWatch holds the potential for broader impact. With minor adaptations, it can be scaled nationally through integration with Department of Health surveillance systems.

DengueWatch exemplifies how deep learning can bridge the gap between data science and life-saving interventions. It empowers health workers to act preemptively, policymakers to allocate resources strategically, and communities to engage in early preventive measures. As climate change accelerates the spread of vector-borne diseases, systems like DengueWatch will become indispensable in safeguarding public health. This system not only demonstrates the power of AI in epidemiological forecasting but also lays the foundation for a smarter, more resilient approach to combating infectious diseases in the years ahead.

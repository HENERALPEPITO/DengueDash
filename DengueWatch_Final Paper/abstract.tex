% Filename    : abstract.tex

\begin{center}
	\textbf{Abstract}
\end{center}

{\small
	\begin{spacing}{1.5}  % Reduce line spacing slightly below single
		Dengue fever remains a significant public health concern in the Philippines, with cases rising dramatically in recent years. Iloilo City experienced a surge in cases, with 4,585 reported cases and 10 deaths as of August 10, 2023, a 319\% increase from the previous year’s 1,095 cases and one death. This rise overwhelmed local healthcare facilities, with over 76\% of non-COVID-19 hospital beds occupied by dengue patients. The lack of a reliable monitoring and forecasting system delayed interventions, worsening the public health burden. To address this, the study developed a centralized system to modernize data management and monitoring of dengue cases in public health institutions. Using data from the Iloilo Provincial Health Office and online sources, several deep learning models were trained to forecast dengue cases on weather variables and historical data. Models tested included LSTM, ARIMA, Seasonal ARIMA, Kalman Filter (KF), and a hybrid KF-LSTM, evaluated with time series cross-validation and error metrics like MSE, RMSE, and MAE. The LSTM model performed best, achieving the lowest RMSE of 20.15, followed by the hybrid KF-LSTM with 25.56. The LSTM model was integrated into the system, providing forecasting capabilities to support proactive interventions and better resource planning in health institutions.
	\end{spacing}
}

\vspace{0.5em}
\begin{tabular}{lp{4.25in}}
	\hspace{-0.5em}\textbf{Keywords:}\hspace{0.25em} & Dengue, Deep Learning, Monitoring, Prediction \\
\end{tabular}

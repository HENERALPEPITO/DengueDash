%   Filename    : abstract.tex 
\begin{abstract}
    In response to a marked rise in dengue cases, Iloilo City and Province are enhancing control measures.
     As of August 10, 2023, the Iloilo Provincial Health Office reported 4,585 cases and 10 fatalities, 
     reflecting a 319\% increase from last year's 1,095 cases and one death. This research includes the development of a centralized system for monitoring and forecasting dengue trends in the Iloilo region.
     This study explores the application 
     of artificial intelligence (AI) for dengue prediction, using a deep learning approach with Long Short-Term Memory (LSTM) networks. 
     The LSTM model is compared with traditional statistical methods, including non-seasonal and seasonal Autoregressive Integrated 
     Moving Average (ARIMA) models and the Kalman Filter for state estimation algorithm in noisy data conditions. Forecasting was based on 
     climate variables such as temperature, rainfall, relative humidity, and previous monthly case counts, with performance evaluated using
      Root Mean Square Error (RMSE). This research, aimed at supporting public health agencies 
      like the Department of Health (DOH), advocates for AI-driven solutions that improve outbreak response beyond traditional reporting systems.


\begin{flushleft}
\begin{tabular}{lp{4.25in}}
\hspace{-0.5em}\textbf{Keywords:}\hspace{0.25em} &  ARIMA, artificial intelligence, dengue prediction, LSTM, Kalman Filter, 
deep learning, climate variables, public health, outbreak mitigation
\\
\end{tabular}
\end{flushleft}
\end{abstract}

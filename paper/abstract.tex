%   Filename    : abstract.tex 
\begin{abstract}
Dengue fever remains a significant public health concern in the Philippines, with cases rising dramatically in recent years. Nationwide outbreaks have placed immense strain on healthcare systems, underscoring the need for innovative approaches to surveillance and response. In Iloilo City, this national trend is reflected in a significant surge, with the Iloilo Provincial Health Office reporting 4,585 cases and 10 fatalities as of August 10, 2023—a 319\% increase from the previous year’s 1,095 cases and one death. This research includes the development of a centralized system for monitoring and forecasting dengue trends in Iloilo City.
This study explores the application of artificial intelligence (AI) for dengue prediction, using a deep learning approach with Long Short-Term Memory (LSTM) networks. The LSTM model is compared with traditional statistical methods, including non-seasonal and seasonal Autoregressive Integrated Moving Average (ARIMA) models and the Kalman Filter for state estimation algorithm in noisy data conditions. Forecasting was based on climate variables such as temperature, rainfall, relative humidity, and previous monthly case counts, with performance evaluated using Root Mean Square Error (RMSE). By integrating predictive analytics with real-time data visualization, the proposed system aims to support public health agencies, such as the Department of Health (DOH), by providing actionable insights for proactive intervention strategies. This AI-driven solution enhances traditional outbreak reporting systems by enabling timely, data-informed decisions to mitigate the impact of dengue in the region.


\begin{flushleft}
\begin{tabular}{lp{4.25in}}
\hspace{-0.5em}\textbf{Keywords:}\hspace{0.25em} &  ARIMA, artificial intelligence, dengue prediction, LSTM, Kalman Filter, 
deep learning, climate variables, public health, outbreak mitigation
\\
\end{tabular}
\end{flushleft}
\end{abstract}

%   Filename    : abstract.tex 
\begin{abstract}
Dengue fever remains a significant public health concern in the Philippines, with cases rising dramatically in recent years. Nationwide outbreaks have placed immense strain on healthcare systems, underscoring the need for innovative approaches to surveillance and response. In Iloilo City, this national trend was reflected in a significant surge, with the Iloilo Provincial Health Office reporting 4,585 cases and 10 fatalities as of August 10, 2023—a 319\% increase from the previous year’s 1,095 cases and one death. This study developed a centralized system for monitoring and modernizing data management of dengue cases in public health institutions, making it more efficient and acceptable. Using data gathered from the Iloilo Provincial Health Office and online sources, several deep learning models were trained to predict dengue cases, utilizing weather variables and historical case data as inputs. These models included Long Short-Term Memory (LSTM), ARIMA, Seasonal ARIMA, Kalman Filter (KF), and a hybrid KF-LSTM model. The models underwent time series cross-validation strategies to mimic real-world conditions as closely as possible and were evaluated using metrics such as Mean Squared Error (MSE), Root Mean Squared Error (RMSE), and Mean Absolute Error (MAE). The LSTM model demonstrated the best performance with the lowest RMSE of 16.90, followed by the hybrid KF-LSTM model at 25.56. The LSTM model was then integrated into the system to provide forecasting features that could support health institutions by offering actionable insights for proactive intervention strategies.

\begin{flushleft}
\begin{tabular}{lp{4.25in}}
\hspace{-0.5em}\textbf{Keywords:}\hspace{0.25em} &  ARIMA, artificial intelligence, dengue prediction, LSTM, Kalman Filter, 
deep learning, climate variables, public health, outbreak mitigation
\\
\end{tabular}
\end{flushleft}
\end{abstract}

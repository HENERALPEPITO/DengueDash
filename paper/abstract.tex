%   Filename    : abstract.tex 
\begin{abstract}
Dengue fever remains a significant public health concern in the Philippines, with cases rising dramatically in recent years. Nationwide outbreaks have placed immense strain on healthcare systems, underscoring the need for innovative approaches to surveillance and response. In Iloilo City, this national trend was reflected in a significant surge, with the Iloilo Provincial Health Office reporting 4,585 cases and 10 fatalities as of August 10, 2023—a 319\% increase from the previous year’s 1,095 cases and one death. This research focused on developing a centralized system for monitoring and forecasting dengue trends in Iloilo City, incorporating graphical visualizations such as heatmaps, trends, and historical graphs. The study explored the application of artificial intelligence (AI) in dengue prediction, utilizing deep learning models. The performance of the Long Short-Term Memory (LSTM) model was compared with traditional statistical methods, including non-seasonal and seasonal Autoregressive Integrated Moving Average (ARIMA) models and the Kalman Filter. Evaluation metrics such as Mean Absolute Error (MAE), Mean Squared Error (MSE), and Root Mean Square Error (RMSE) were used to identify the most effective model for integration into the system. Forecasting was based on climate variables such as temperature, rainfall, relative humidity, and previous monthly case counts. The LSTM model emerged as the best performer with an RMSE of 16.15, demonstrating its suitability for time-series predictions, while the Kalman Filter showed the poorest performance with an RMSE of 38.40. By integrating predictive analytics with real-time data visualization, the proposed system aims to support public health agencies, such as the Department of Health (DOH), by providing actionable insights for proactive intervention strategies. This AI-driven solution enhances traditional outbreak reporting systems by enabling timely, data-informed decisions to mitigate the impact of dengue in the region.

\begin{flushleft}
\begin{tabular}{lp{4.25in}}
\hspace{-0.5em}\textbf{Keywords:}\hspace{0.25em} &  ARIMA, artificial intelligence, dengue prediction, LSTM, Kalman Filter, 
deep learning, climate variables, public health, outbreak mitigation
\\
\end{tabular}
\end{flushleft}
\end{abstract}

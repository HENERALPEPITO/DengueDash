%   Filename    : chapter_2.tex 
\chapter{Review of Related Literature}
\label{sec:relatedlit}

\begin{comment}
A literature review is a piece of discursive prose, not a list describing or summarizing one piece of literature after another. It’s usually a bad sign to see every paragraph beginning with the name of a researcher. Instead, organize the literature review into sections that present themes or identify trends, including relevant theory. You are not trying to list all the materials published, but to synthesize and evaluate them according to the guiding concept of your thesis or research question. You should also state the limits or gaps of their researches wherein you will try to fill these gaps in accordance to your research problem and objectives.	content...
\end{comment}



\begin{comment}

%
% IPR acknowledgement: the contents withis this comment are from Ethel Ong's slides on RRL.
%
Guide on Writing your RRL chapter

1. Identify the keywords with respect to your research
One keyword = One document section
Examples: 2.1 Story Generation Systems
2.2 Knowledge Representation

2.  Find references using these keywords

3.  For each of the references that you find,
Check: Is it relevant to your research?
Use their references to find more relevant works.

4. Identify a set of criteria for comparison.
It will serve as a guide to help you focus on what to look for

5. Write a summary focusing on -
What: A short description of the work
How: A summary of the approach it utilized
Findings: If applicable, provide the results
Why: Relevance to your work

6. At the end of each section,  show a Table of Comparison of the related works
and your proposed project/system

\end{comment}

\section{Existing System: RabDash DC}
%
% IPR acknowledgement: the following list of items are from Ethel Ong's slides RRL.
%
RabDash, developed by the University of the Philippines Mindanao, is a web-based dashboard for rabies data analytics. It combines predictive modeling with genomic data, enabling local health authorities to optimize interventions and allocate resources more effectively. RabDash's modules include trend visualization, geographic hotspot mapping, and predictive forecasting, utilizing Long Short-Term Memory (LSTM) models for time-series forecasting \cite{rabdash}.

For DengueDash, RabDash serves as a strong inspiration, particularly in its monitoring, historical trend visualization, and forecasting capabilities. These features align well with the needs of dengue control efforts, providing real-time insights into outbreak trends and enabling more effective, data-driven decision-making. RabDash’s architecture is relevant to the DengueDash, as dengue outbreaks similarly require time-series forecasting models. By using LSTM, RabDash effectively models trends in outbreak data, which provides a framework for adapting LSTM to dengue forecasting. Research indicates that LSTM models outperform traditional methods, such as ARIMA and MLP, in handling the complexities of time-dependent epidemiological data \cite{ligue2022deep}.


\section{Deep Learning}
The study of Kim Dianne Ligue and Kristine Joy Ligue highlights how data-driven models can help predict dengue outbreaks. The authors compared traditional statistical methods, such as non-seasonal and seasonal autoregressive integrated moving average (ARIMA), and traditional feed-forward network approach using a multilayer perceptron (MLP) model with a deep learning approach using the long short-term memory (LSTM) architecture in their prediction model. They find that the LSTM model performs better in terms of accuracy. The LSTM model achieved a much lower root mean square error (RMSE) compared to both MLP and ARIMA models, proving its ability to capture complex patterns in time-series data \cite{ligue2022deep}. This superior performance is attributed to LSTM’s capacity to capture complex, time-dependent relationships within the data, such as those between temperature, rainfall, humidity, and mosquito populations, all of which contribute to dengue incidence \cite{ligue2022deep}.

\section{Kalman Filter}
The Kalman Filter is another powerful tool for time-series forecasting that can be integrated into our analysis. It provides a recursive solution to estimating the state of a linear dynamic system from a series of noisy measurements. Its application in epidemiological modeling can enhance prediction accuracy by accounting for uncertainties in the data\cite{li2022applications}. Studies have shown that Kalman filters are effective in predicting infectious disease outbreaks by refining estimates based on observed data. A study published in Frontiers in Physics utilized the Kalman filter to predict COVID-19 deaths in Ceará, Brazil. They found that the Kalman filter effectively tracked the progression of deaths and cases, providing critical insights for public health decision-making \cite{ahmadini2021analysis}. Another research article in PLOS ONE focused on tracking the effective reproduction number ($R_t$) of COVID-19 using a Kalman filter. This method estimated the growth rate of new infections from noisy data, demonstrating that the Kalman filter could maintain accurate estimates even when case reporting was inconsistent\cite{arroyo2021tracking}.

Our study will compare ARIMA, seasonal ARIMA, Kalman Filter, and LSTM models using our own collected dengue case data along with weather data to identify the most effective model for real-time forecasting.

\section{Weather Data}
The relationship between weather patterns and mosquito-borne diseases is inherently nonlinear, meaning that fluctuations in disease cases do not respond proportionally to changes in climate variables\cite{colon2013effects} Weather data, such as minimum temperature and accumulated rainfall, are strongly linked to dengue case fluctuations, with effects observed after several weeks due to mosquito breeding and virus incubation cycles. Integrating these lagged weather effects into predictive models can improve early warning systems for dengue control\cite{cheong2013assessing}. A study also suggests that weather-based forecasting models using variables like mean temperature and cumulative rainfall can provide early warnings of dengue outbreaks with high sensitivity and specificity, enabling predictions up to 16 weeks in advance\cite{hii2012forecast}.

We will utilize weather data, including variables such as temperature, rainfall, and humidity, as inputs for our dengue forecasting model. Given the strong, nonlinear relationship between climate patterns and dengue incidence, these weather variables, along with their lagged effects, are essential for enhancing prediction accuracy and providing timely early warnings for dengue outbreaks.


\section{Chapter Summary}
This chapter reviewed key literature relevant to our study, focusing on existing systems, predictive modeling techniques and the role of weather data in forecasting dengue outbreaks. We examined systems like RabDash DC, which integrates predictive modeling with real-time data to inform public health decisions, providing a foundational structure for our Dengue Watch System. Additionally, deep learning approaches, particularly Long Short-Term Memory (LSTM) networks, were highlighted for their effectiveness in time-series forecasting, while alternative methods such as ARIMA and Kalman Filters were considered for their ability to model complex temporal patterns and handle noisy data.

The literature further underscores the significance of weather variables—such as temperature and rainfall—in forecasting dengue cases. Studies demonstrate that these variables contribute to accurate outbreak prediction models. Leveraging these insights, our study will incorporate both weather data and historical dengue case counts to build a reliable forecasting model.








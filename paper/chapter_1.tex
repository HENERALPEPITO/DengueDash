%   Filename    : chapter_1.tex 
\chapter{Introduction}
\label{sec:researchdesc}    %labels help you reference sections of your document

\section{Overview}
\label{sec:overview}

From 2020 to 2022, dengue cases declined due to reduced surveillance during the COVID-19 pandemic \cite{WHO2023}, but cases surged in 2023 as restrictions were lifted. This year saw an increase in dengue outbreaks worldwide, with over five million cases and more than 5,000 deaths reported in over 80 countries \cite{bosano2023who}.  Dengue is endemic in the Philippines, leading to longer and more widespread seasonal outbreaks. Globally, dengue infections have increased significantly, posing a major public health challenge. The World Health Organization reported a ten-fold rise in cases between 2000 and 2019, with a peak in 2019 when the disease spread across 129 countries \cite{WHO2024}.

Iloilo City and Province are intensifying efforts to curb the rising dengue cases \cite{PNA2024}. As of August 10, 2023, the Iloilo Provincial Health Office recorded 4,585 cases and 10 deaths, a 319\% increase from last year’s 1,095 cases and one death. Governor Arthur Defensor Jr. confirmed that the province has reached the dengue outbreak threshold based on Department of Health (DOH). Local government units (LGUs) have been informed, and the province's disaster management office is on blue alert, indicating disaster mode. \cite{lena2024}

In Iloilo City, 649 dengue cases were recorded this year 2024, with two deaths. Cases cluster in 40 out of 180 barangays, meaning multiple cases are being reported in these areas over several weeks. The city’s health officer, Dr. Roland Jay Fortuna, reported high utilization of non-COVID-19 hospital beds, reaching over 76\%, prompting concerns about hospital capacity.  

This study explores the monitoring and forecasting of dengue outbreaks by analyzing key factors such as temperature, relative humidity, and historical dengue cases, using different models. The findings aim to provide an advanced, AI-driven alternative for dengue prevention and control, targeting agencies like the Department of Health (DOH). By aligning with the national AI Roadmap, particularly in Iloilo City, this research aspires to improve outbreak responses through cutting-edge technology rather than traditional reporting methods.

\section{Problem Statement}
Dengue remains a critical public health challenge worldwide, with cases increasing due to the easing of COVID-19 restrictions and heightened global mobility. While a temporary decline in cases was observed during the pandemic (2020–2022) due to reduced surveillance efforts, 2023 marked a resurgence, with over five million cases and more than 5,000 deaths reported across 80 countries. In dengue-endemic regions like the Philippines, the threat is particularly severe. In Iloilo City and Province, dengue cases rose by 319\% as of August 2023, overwhelming local healthcare systems. This surge strained resources, with over 76\% of non-COVID-19 hospital beds occupied by dengue patients, highlighting the urgent need for effective predictive tools. The lack of a reliable system to monitor and forecast dengue outbreaks contributes to delayed interventions, exacerbating public health risks and healthcare burdens in the region. 

\section{Research Objectives}
\label{sec:researchobjectives}

\subsection{General Objective}
\label{sec:generalobjective}

This study aims to develop an AI-based dengue forecasting and monitoring system for Iloilo City and Province. 
The researchers will train and compare multiple deep learning models to predict dengue case trends based on climate data and 
historical dengue cases to help public health officials in possible dengue case outbreaks.


\subsection{Specific Objectives}
\label{sec:specificobjectives}

%
%  \begin{comment} ... \end{comment} is used for multiple lines of comment
%

Specifically, this study aims to:


\begin{enumerate}
	\item Gather dengue data from the Iloilo Provincial Health Office and climate data from online sources. Combine and aggregate these data into a unified dataset to facilitate comprehensive dengue case forecasting.
	\item Evaluate deep learning models for predicting dengue cases. Compare the performance of these models to determine the most accurate forecasting approach.
	\item Develop a web-based analytics dashboard that integrates a predictive model and provides data management  system for dengue cases in  Iloilo City and the Province.
\end{enumerate}


\section{Scope and Limitations of the Research}
\label{sec:scopelimitations}

This study aims to gather dengue data from the Iloilo Provincial Health Office and climate data from online sources such as PAGASA or weatherandclimate.com. These data will be preprocessed, cleaned, and combined into a unified dataset to facilitate comprehensive dengue case forecasting. However, the study is limited by the availability and completeness of historical data. Inconsistent or missing data points may introduce biases and reduce the quality of predictions. Furthermore, the granularity of the data will be in a weekly format.

To evaluate deep learning models for predicting dengue cases, the study will train and compare the performance of various models, using metrics like Mean Absolute Error (MAE) and Root Mean Square Error (RMSE). While these models aim to provide accurate forecasts, their performance is heavily influenced by the quality and size of the dataset. Limited or low-quality data may lead to suboptimal predictions. Additionally, the models cannot fully account for external factors such as public health interventions or socio-economic conditions which may impact dengue transmission dynamics.

The study also involves developing a web-based analytics dashboard that integrates predictive models and provides a data management system for dengue cases in Iloilo City and the Province. This dashboard will offer public health officials an interactive interface to visualize dengue trends, input new data, and identify risk areas. However, its usability depends on feedback from stakeholders, which may vary based on their familiarity with analytics tools. Moreover, external factors such as limited internet connectivity or device availability in remote areas may affect the system's adoption and effectiveness. While the dashboard provides valuable insights, it cannot incorporate all factors influencing dengue transmission, emphasizing the need for ongoing validation and refinement.


\section{Significance of the Research}
\label{sec:significance}

This study’s development of an AI-based dengue forecasting and monitoring system has wide-reaching significance for various stakeholders in Iloilo City:

\begin{itemize}
	\item  Public Health Agencies: Organizations like the Department of Health (DOH) and local health units in Iloilo City and Province stand to benefit greatly from the system. With dengue predictions, we can help these agencies optimize their response strategies and implement targeted prevention measures in high-risk areas before cases escalate.
\end{itemize}

\begin{itemize} 
	\item Local Government Units (LGUs): LGUs can use the system to support their disaster management and health initiatives by proactively addressing dengue outbreaks. The predictive insights allow for more efficient planning and resource deployment in barangays and communities most vulnerable to outbreaks, improving overall public health outcomes.
\end{itemize}


\begin{itemize} 
	\item Healthcare Facilities: Hospitals and clinics, which currently face high bed occupancy rates during dengue season will benefit from early outbreak forecasts that can help in managing patient inflow and ensuring adequate hospital capacity. 
\end{itemize}


\begin{itemize} 
	\item Researchers and Policymakers: This AI-driven approach contributes valuable insights for researchers studying infectious disease patterns and policymakers focused on strengthening the national AI Roadmap. The system's data can support broader initiatives for sustainable health infrastructure and inform policy decisions on resource allocation for dengue control.
\end{itemize}


\begin{itemize} 
	\item Community Members: By reducing the frequency and severity of outbreaks, this study ultimately benefits the community at large. This allows for timely awareness campaigns and community engagement initiatives, empowering residents with knowledge and preventative measures to protect themselves and reduce the spread of dengue.
\end{itemize}


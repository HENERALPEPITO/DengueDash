%   Filename    : chapter_1.tex 
\chapter{Introduction}
\label{sec:researchdesc}    %labels help you reference sections of your document

\section{Overview}
\label{sec:overview}

From 2020 to 2022, dengue cases declined due to reduced surveillance during the COVID-19 pandemic, but cases surged in 2023 as restrictions were lifted. This year saw an increase in dengue outbreaks worldwide, with over five million cases and more than 5,000 deaths reported in over 80 countries. \cite{bosano2023who} Dengue is endemic in the Philippines, leading to longer and more widespread seasonal outbreaks. Globally, dengue infections have increased significantly, posing a major public health challenge. The World Health Organization \cite{WHO2024} reported a ten-fold rise in cases between 2000 and 2019, with a peak in 2019 when the disease spread across 129 countries.

Iloilo City and Province are intensifying efforts to curb the rising dengue cases. As of August 10, 2023, the Iloilo Provincial Health Office recorded 4,585 cases and 10 deaths, a 319\% increase from last year’s 1,095 cases and one death. Governor Arthur Defensor Jr. confirmed that the province has reached the dengue outbreak threshold based on Department of Health (DOH) criteria, and a formal declaration is pending. Local government units (LGUs) have been informed, and the province's disaster management office is on blue alert, indicating disaster mode. \cite{lena2024}

In Iloilo City, 649 dengue cases were recorded during the same period, with two deaths. Cases cluster in 40 out of 180 barangays, meaning multiple cases are being reported in these areas over several weeks. The city’s health officer, Dr. Roland Jay Fortuna, reported high utilization of non-COVID-19 hospital beds, reaching over 76\%, prompting concerns about hospital capacity.  
\\This study explores the monitoring and forecasting of dengue outbreaks by analyzing key factors such as temperature, relative humidity, and historical dengue cases, using different models. The findings aim to provide an advanced, AI-driven alternative for dengue prevention and control, targeting agencies like the Department of Health (DOH). By aligning with the national AI Roadmap, particularly in Iloilo City, this research aspires to improve outbreak responses through cutting-edge technology rather than traditional reporting methods.

\section{Problem Statement}
The problem being addressed here is that dengue cases remain a critical public health issue worldwide,
 with rising cases attributed to the easing of COVID-19 restrictions and increased global mobility. 
 From 2020 to 2022, dengue cases saw a temporary decline due to reduced surveillance efforts amidst the pandemic. 
 However, 2023 witnessed a significant resurgence, with over five million cases and more than 5,000 deaths reported 
 across 80 countries, indicating the continued vulnerability of dengue-endemic regions like the Philippines. 
 In Iloilo City and Province, dengue cases surged dramatically by 319\% as of August 2023, 
 with local health systems struggling to manage the influx. High hospitalization rates due to dengue, with over
  76\% of non-COVID-19 hospital beds occupied, have raised concerns about healthcare capacity and the need for enhanced predictive measures. 

\section{Research Objectives}
\label{sec:researchobjectives}

\subsection{General Objective}
\label{sec:generalobjective}

This study aims to develop an AI-based dengue forecasting and monitoring system for Iloilo City and Province. 
The system will use Long Short-Term Memory (LSTM) to predict dengue case trends based on climate data and 
historical dengue cases to help public health officials in possible dengue case outbreaks.


\subsection{Specific Objectives}
\label{sec:specificobjectives}

%
%  \begin{comment} ... \end{comment} is used for multiple lines of comment
%

Specifically, this study aims to develop a system that can:


\begin{enumerate}
	\item Gather dengue data from the Iloilo Provincial Health Office and climate data from online sources. Combine these data into a unified dataset to facilitate comprehensive dengue case forecasting.
	\item Develop and evaluate deep learning models, including LSTM, ARIMA, Seasonal ARIMA, and Kalman Filter, for predicting dengue cases. Compare the performance of these models to determine the most accurate forecasting approach.
	\item Integrate the predictive model into a web-based data analytics dashboard. This dashboard will include features such as data visualizations and data entry, offering public health stakeholders an interactive tool for analyzing dengue trends and making informed decisions.
\end{enumerate}


\section{Scope and Limitations of the Research}
\label{sec:scopelimitations}

This study aimed to develop an AI-based dengue forecasting and monitoring system specifically designed for Iloilo City. The system focuses on two major features: dengue case prediction and risk area identification. The dengue case prediction feature utilizes climate variables—such as temperature, rainfall, and relative humidity—along with historical dengue case data to forecast monthly dengue cases. The results will be displayed in a user-friendly interface, providing public health officials with actionable insights to enhance outbreak management and resource allocation.
However, this study has several limitations. The accuracy of the dengue case predictions heavily relies on the quality and completeness of the input data. Inconsistent or incomplete historical data may lead to reduced prediction accuracy. Additionally, the model’s performance may fluctuate based on variations in climate patterns, which are not always predictable.
The model utilizes advanced machine learning techniques, but it cannot account for all factors influencing dengue transmissions, such as socio-economic conditions or public health interventions, which may further impact case dynamics.
Finally, the dataset used for training the predictive models has not undergone peer review but has been validated by local public health experts to ensure its relevance and accuracy for the study's context. As a result, the findings should be interpreted with caution, and ongoing validation and adjustments may be necessary to enhance the model's robustness and applicability in real-world settings.


\section{Significance of the Research}
\label{sec:significance}

This study’s development of an AI-based dengue forecasting and monitoring system has wide-reaching significance for various stakeholders in Iloilo City:

\begin{itemize}
\item  Public Health Agencies: Organizations like the Department of Health (DOH) and local health units in Iloilo City and Province stand to benefit greatly from the system. With dengue predictions, we can help these agencies optimize their response strategies and implement targeted prevention measures in high-risk areas before cases escalate.
\end{itemize}

\begin{itemize} 
\item Local Government Units (LGUs): LGUs can use the system to support their disaster management and health initiatives by proactively addressing dengue outbreaks. The predictive insights allow for more efficient planning and resource deployment in barangays and communities most vulnerable to outbreaks, improving overall public health outcomes.
\end{itemize}


\begin{itemize} 
\item Healthcare Facilities: Hospitals and clinics, which currently face high bed occupancy rates during dengue season will benefit from early outbreak forecasts that can help in managing patient inflow and ensuring adequate hospital capacity. 
 \end{itemize}


\begin{itemize} 
\item Researchers and Policymakers: This AI-driven approach contributes valuable insights for researchers studying infectious disease patterns and policymakers focused on strengthening the national AI Roadmap. The system's data can support broader initiatives for sustainable health infrastructure and inform policy decisions on resource allocation for dengue control.
\end{itemize}


\begin{itemize} 
\item Community Members: By reducing the frequency and severity of outbreaks, this study ultimately benefits the community at large. This allows for timely awareness campaigns and community engagement initiatives, empowering residents with knowledge and preventative measures to protect themselves and reduce the spread of dengue.
\end{itemize}

